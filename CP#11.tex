\documentclass[10pt]{article}

\usepackage{amsmath}

\begin{document}
	\title{Respuesta de la Clase Pr\'actica 11}
	\author{Daniel De La Cruz Prieto}
	\date{\today}
	\maketitle
	
	\section*{Respuesta del ejercicio 1.}
    \begin{flushleft}
    	El ancho de la banda que utiliza un usuario de la univercidad es una variable aleatoria independiente para cada usuario y con la misam distribuci\'on . Se conoce que su valor esperado es 1Kb/s y la desviaci\'on estandar es 0.5Kb/s . suponga que hay 100 usuarios conectados en un momento determinado . Calcula la probabilidad de que el eancho de banda que se est\'a utilizando sobrepase los 96 Kb/s
    \end{flushleft}
    
    \begin{flushleft}
     	Por lo tant	tenemos que : $ n = 100 $  \hspace{1cm} $ \sigma = \sqrt{V(x) = 0.5}$
    \end{flushleft}
    
    \begin{flushleft}
	    Se quiere hallar :
    \end{flushleft}
    
   	\begin{equation*}
   	    P \left(\sum_{i=1}^{100} X_{i} = 96\right)
   	\end{equation*}
   	
    \begin{flushleft}
   	   Se supone la hip\'otesis del teorema central del l\'imite $ \left(Linderbeg - Levy\right)$ :
    \end{flushleft}
   	
   	\begin{equation*}
   	   \displaystyle\frac{\sum_{i=1}^{n} X_i -nu}{\sqrt{n}\sigma} \longrightarrow Z\sim N (0,1)
   	\end{equation*}
   	
    \begin{flushleft}
   	    Por lo tanto podemos decir :
    \end{flushleft}
   	\begin{equation*}
		\begin{array}{rcl}
	   	    \displaystyle P\left(\sum_{i=1}^{100}X_{i} \leq 96\right) & = & 1 -  P\left(X \leq  96\right) 
	   	    \\
	   	    \\
	   	    & = &  \displaystyle P\left(\frac{\sum_{i=1}^{100}X_{i} - 100}{\sqrt{100}* \frac{1}{2}} \leq \frac{96 - 100}{\sqrt{100}* \frac{1}{2}}\right) 
	   	    \\
	   	    \\
	   	    & = &  \displaystyle 1 - P\left(Z \leq -\frac{4}{5}\right)Z \sim N(0,1)
	   	    \\
	   	    \\
	   	    & = &  \displaystyle 1 - P\left(Z \geq \frac{4}{5}\right)
	   	    \\
	   	    \\
	   	    & = &  \displaystyle 1 -\left( 1 - P\left(Z \leq \frac{4}{5}\right)\right)
	   	    \\
	   	    \\
	   	    & = &  \displaystyle P\left(Z \leq \frac{4}{5}\right)
	   	    \\
	   	    \\
	        & = & 0.7881
	    \end{array}
	\end{equation*}
	   
	\begin{flushleft}
		Entonces podemos concluir que la probabilidad de que el ancho de banda sobrepase los $96$  kb/s es de aproximadamente  $0.7881$
	\end{flushleft}

  	\section*{Respuesta del ejercicio 2.} 
   	\begin{flushleft}
   		El tama\~no de un aula de primer a\~no de la universidad es de $150$ estudiantes . La direcci\'on de la universidad sabe de experiencias anteriores que en promedio solo el 30 \% de los aceptados por la univesidad realmente asistiran por lo que utiliza la pol\'itica de aprobar 450 solicitudes de admisi\'on . Calcula  la probabilidad de que m\'as de 150 estudiantes de primer a\~no asistan a esta universidad .
   	\end{flushleft}
   
   	\begin{flushleft}
   	   	$\to$ Tenemos $ E(X_k) = p = 0.3 $ y $V(X_k) = p\left(1-p\right) = 94.5$
    \end{flushleft}
    
    \begin{flushleft}
   		Se cumple la hip\'otesisi del teorema central del l\'imite (Moivre - Laplace)
   	\end{flushleft}
   	
    \begin{equation*}
	   	\displaystyle\frac{\sum_{i=1}^{n} X_i -np}{\sqrt{np\left(1-p\right)}} \longrightarrow Z\sim N (0,1)
	\end{equation*}
    \begin{flushleft}
	   	Por lo tanto podemos decir : 
    \end{flushleft}
   	\begin{equation*}
	   	\begin{array}{rcl}
	   		\displaystyle P\left(\sum_{i=1}^{450}X_{i} > 150 \right) & = & \displaystyle P\left(\frac{\sum_{i=1}^{450}X_{i} - 450 * 0.3}{\sqrt{ 450 * (0.3)(1-(0.3)}}  > \frac{150 - 450 * (0.3)}{\sqrt{(450 * 0.3)*(1- 0.3)}}\right) 
	   	  	\\
	   	    \\
			   & = & \displaystyle P\left(\frac{\sum_{i=1}^{450}X_{i} - 135}{\sqrt{135(0.7)}}  > \frac{150 - 135}{(135)(0.7)}\right) 
	   		\\
	   		\\
			& = & 	\displaystyle P\left(Z > \frac{15}{\sqrt{94.5)}}\right) 
			\\
			\\
			& = & P\left(Z > 1.543\right)
			\\
			\\
			& = & 1 - P\left(Z \leq 1.543\right)
			\\
			\\
			& = & 1 - 0.9382
			\\
			\\
			& = &0.0618
	   	\end{array}
	\end{equation*}
	
	\begin{flushleft}
		Entonces podemos decir que la probabilidad de que mas de $150$ estudiantes de primer a\~no asistan a la universidad es de aproximadamente $0.0618$
	\end{flushleft}
	   
	
	\section*{Ejercicio 3 } 

	\begin{flushleft}
		Se tiene una caja con 3 bolas blancas y 2 bolas negras y se toma al azar una bola, despu\'es de anotar el color se devuelve a la caja. Si este experimento se repite 1000 veces de forma independiente, ¿cu\'al es la probabilidad de que al menos 410 bolas hayan sido bolas negras?
	\end{flushleft}

	\begin{flushleft}
		Tenemos $n=1000$ $p=\frac{2}{5}=0.4$ $E(X_k) = p=0.4$ y $V (X_k) = p(1- p)=0.24$
	\end{flushleft}
	
	\begin{flushleft}
		Se cumple la hip\'otesis del Teorema Central del L\'imite (Moivre-Laplace):
	\end{flushleft}


	\begin{equation*}
		\frac{\sum_{i=1}^{n}X_i-np}{\sqrt{np(1-p)}}
	\end{equation*}


	\begin{flushleft}
		Entonces vamos a calcular la probabilidad , para esto tenemos : 
	\end{flushleft}
	
	\begin{equation*}
		\begin{array}{rcl}
			P(\sum_{i=1}^{1000}X_i > 410)  	& = &P(\frac{\sum_{i=1}^{1000}X_i-400}{\sqrt{240}}> \frac{410-400}{\sqrt{240}})
			\\
			\\
											& = & P(\frac{\sum_{i=1}^{1000}X_i-600}{\sqrt{240}}> -\frac{19\sqrt{15}}{6})
			\\
			\\
											& = &P(Z>0.64)
			\\
			\\
											& = & 1-P(Z<0.64)
			\\
			\\
	                                         & = & 1-0.7389
			\\
			\\
											 &\approx &  0.2611
			\\
			\\
		\end{array}
	\end{equation*}

	\begin{flushleft}
		Por lo tanto podemos decir que la probabilidad de que al menos $410$ bolas hayan sido negra es aproximadamente  $0.2611$
	\end{flushleft}

	

\end{document}	